\nonstopmode % halt on errors
\documentclass[onecolumn, draftclsnofoot,10pt, compsoc]{IEEEtran}
\usepackage{graphicx}
\usepackage{url}
\usepackage{setspace}

\usepackage{geometry}
\geometry{textheight=9.5in, textwidth=7in}


% 1. User headings
\def \ClassName {CS 444}
\def \AuthorName {Scott Merrill and Drake Seifert}
\def \DocName {Homework Assignment \#3}


% 2. Code markup

\usepackage{xcolor}
\usepackage{listings}

\definecolor{mGreen}{rgb}{0,0.6,0}
\definecolor{mGray}{rgb}{0.5,0.5,0.5}
\definecolor{mPurple}{rgb}{0.58,0,0.82}
\definecolor{backgroundColour}{rgb}{0.95,0.95,0.92}

\lstdefinestyle{CStyle}{
    backgroundcolor=\color{backgroundColour},   
    commentstyle=\color{mGreen},
    keywordstyle=\color{magenta},
    numberstyle=\tiny\color{mGray},
    stringstyle=\color{mPurple},
    basicstyle=\footnotesize,
    breakatwhitespace=false,         
    breaklines=true,                 
    captionpos=b,                    
    keepspaces=true,                 
    numbers=left,                    
    numbersep=5pt,                  
    showspaces=false,                
    showstringspaces=false,
    showtabs=false,                  
    tabsize=2,
    language=C
}

\begin{document}
\begin {titlepage}
	\pagenumbering{gobble}
    \begin{singlespace}
    	\hfill
        \par\vspace{.2in}
    	\centering
    	\scshape{
    		\huge \DocName \par
            \par\vspace{.2in}
            \large \ClassName \par
            
            \par\vspace{7in}
            \large By: \AuthorName\par
            \par\vspace{.2in}
            \small Abstract: This assignment is designed to have us create a simple I/O block device and implement basic encryption and decryption schemes. 
    	}
    \end{singlespace}
\end{titlepage}


\section{Design Plan}
To implement our driver we used the simple block driver design from http://blog.superpat.com/2010/05/04/a-simple-block-driver-for-linux-kernel-2-6-31/. This resource provides a simple algorithm for how to implement a basic block driver. It consists of essentially four functions: sbd\_request(), sbd\_transfer(), and \_init(), and \_exit(). The init function is pretty simple as it just initializes the driver. The request function checks the request queue and then passes into the transfer fuction which handles the I/O request that was in the request queue. For I/O requests we wanted to have a way to handle both read and write requests. Within the read request we needed to have some basic cryptology that will decrypt the file using a given key. When handling write requests we wanted to have the block driver encrypt the data before writing to the file. Lastly the exit fuction removes and cleans up all allocated memory. 

\section{Questions}
\begin{enumerate}
\item \textbf{What do you think the main point of this assignment is?}
\\	The main point of this assignment was to become more familiar with how the kernel handles modules. Specifically by implementing a block device for I/O we can expand on what we learned in assignment 2 when dealing with process requests. By including the requirement for cryptology we also gain a basic understanding of how cryptology can be used by the kernel to protect files. 

\item \textbf{How did you personally approach the problem? Design decisions, algorithm, etc.}
\\ The hardest part of this assignment was figuring out where to begin. We began by doing research into how to create a block device, how to create file system and how to transfer the files into the virtual machine. We were directed to a source that provided a simple block device implementation "http://blog.superpat.com/2010/05/04/a-simple-block-driver-for-linux-kernel-2-6-31/" that follows the Linux Device Drivers book. It took us a while to discover how to use scp to transfer files into the VM, but it turned out to be some simple qemu command line flags that needed to be set. Lastly we needed to apply crypto to the read and write functionality of the block device. Linux have a useful crypto-library (linux/crypto.h) which we allowed us to implement the encryption before writing, and decryption before reading. The rest was providing step by step instructions on how to prove that the encryption worked.   

\item \textbf{How did you ensure your solution was correct? Testing details, for instance.}
\\ Print statements was the simplest way to show that our implementation worked. We used print statements to show that our block device was working once mounted. We also used print statements to show when the encryption and decryption process was happening and if it was successful or not. Lastly we printed out the encrypted data values to show that encryption happened.  

\item \textbf{What did you learn?}
\\ This assignment furthered our understanding of how I/O requests are handled. We examined how a basic block device is implemented and have gained some insight into how drivers are used to execute code during runtime. This assignment also had us demonstrate how the kernel might use cryptology to provide a measure of security. This allowed up to become familiar with the basic crypto library provided by Linux. 

\item \textbf{How should the TA evaluate your work? Provide detailed steps to prove correctness.} 	\begin{enumerate}
	\item Before setting up the kernel you need to make sure you have the run\_qemu script provided (which changes the command line flags to run our scheduler) as well as our makefile, sbd.c, and Kconfig files that are located in the driver/block/ folder. You also need the core-image-lsb-skd-qemux86.ext4 which is not included in the patch file because it is too big.
    
	\item First you need to set up the environment variables by running the command source environment-setup-i586.pokey-linux.csh
    
    \item Make sure you are using the correct version of the kernel: git checkout 'v3.19.2'
    \item Next you will want build the kernel: make -j4 all
    \item Once the kernel has been compiled you will run the run\_qemu command, which will start the program.
    \item If you are prompted for the status of the sbd driver select 'm'
    \item login as root
    \item run: scp \textbf{onid-username}@os2.oregonstate.edu:\textbf{path-to-sbd.ko} .
    
    \item To test mount the driver:
    \subitem insmod sbd.ko
    \subitem shred -z /dev/sbd0
    \subitem mkfs.ext2 /dev/sbd0
    \subitem mount /dev/sbd0 /mnt/
    \item run ``lsmod" to confirm driver
    
    \item To test:
    \subitem create file: touch testfile
    \subitem write to file: echo "Hello World" \textgreater  testfile
    \subitem read file: cat testfile 
    
    \item To unmount module:
    \subitem umount /mnt
    \subitem lsmod 
	\end{enumerate}
\end{enumerate}

\section{Work and Control log}
\begin{center}
    \begin{tabular}{| p{2.5cm} | l | l | p{3cm} |}    
    	\hline
        Date & Author & Commit & Summary \\ \hline
        Nov 09 2017 & Drake & 8a058ef7496956498b2e081cc55d82e4f19b4dba & Created hw3 directory \\ \hline
        Nov 09 2017 & Scott & 4de332e6c675c39886b461a71efc25755376c722 & added sbd.c file to the driver/block/ directory \\ \hline
        Nov 09 2017 & Drake & 92ed71095b48f4a41f7177a24ced2ee72ff531a2 & Added sbd to Kconfig and makefile \\ \hline
        Nov 12 2017 & Drake & c21dd78d050a3ea51f90dc02e9c4abf4251a09f8 & Added helpful scripts \\ \hline
        Nov 15 2017 & Drake & 78839752fd66b23bf68183d641afca879a408f47 & Enabled networking \\ \hline
        Nov 15 2017 & Drake & d12297b6d648b213fdf3bd0e412e782e834c6ae6 & Minor Updates \\ \hline
        Nov 15 2017 & Scott & 275ae9f08ac39a1599f07789de8b2ffe4fc7818a & Removed broken files \\ \hline
        Nov 15 2017&Drake&31299c505bdacbfb9cd3335f95b8febe35b626a5 & Working version of sbd w/o cipher  \\ \hline     	
	\end{tabular}
\linebreak

\textbf{Work Log:}  
\end{center}
Like in the past assignments most of our coding was done as pair-programming where one would be the pilot and the other would navigate. This allow both of us to gain an understanding of all parts of the assignment. When not working together Scott focused on the write-up, networking for scp, and implemented the first version of the cipher; Drake worked on the first implementation of the sbd.c file, Kconfig, Makefile, as well as ensuring correctness with the cryptology.  

\end{document}